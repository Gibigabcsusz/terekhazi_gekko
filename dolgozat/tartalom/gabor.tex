A feladat megoldásához az $\vec{A}-\Phi, \vec{A}$ formalizmust használjuk, mert az előadáson bemutatott indukciós főzőlapos példa alapján ezzel a megközelítéssel egy jól kezelhető parciális differenciálegyenlet-rendszert kapunk, amit a PDEtool segítségével megoldhatunk.
\begin{align}
    \vec{B} = \rot \vec{A}
\end{align}
A peremfeltételekről a következőket lehet tudni:
\begin{align}
    B_n(\vec{r}) &= 0, \quad \vec{r} \in \Gamma_Z\\
    B_n(\vec{r}) &= 0, \quad \vec{r} \in \Gamma_L \cup \Gamma_T
\end{align}
A $B_n$ normális mágneses indukció a $\Gamma_Z$ peremen a szimmetria miatt lesz 0, a $\Gamma_L \cup \Gamma_T$ távoli peremeken pedig a nagy távolság miatt lesz közel 0, amit pontosan 0 értékkel modellezünk.
\section{A PDE és megadása pdetool-ban}
A tartományokra vonatkozó parciális differenciálegyenlet:
\begin{equation}
    -\left(\dfrac{\partial}{\partial r}\dfrac{1}{r\mu}\dfrac{\partial(rA_{\varphi})}{\partial r}+\dfrac{\partial}{\partial z}\dfrac{1}{r\mu}\dfrac{\partial(rA_{\varphi})}{\partial z}\right) + j\omega\sigma A_{\varphi} = J_{i,\varphi}
\end{equation}
Ezt a következő helyettesítésekkel lehet MATLAB-ban (pdetool-ban) megadni:
\begin{equation}
    \begin{aligned}
        rA_{\varphi}\quad\rightarrow&\quad\verb|u|\\
        r\quad\rightarrow&\quad\verb|x|\\
        z\quad\rightarrow&\quad\verb|y|
    \end{aligned}
\end{equation}
A használt elliptikus PDE séma a következő:
\begin{equation}
    \verb|-div(c*grad(u))+a*u=f|
\end{equation}
Eszerint az egyes paraméterek behelyettesítési értékei:
\begin{equation}
    \begin{aligned}
        \verb|c|\quad=&\quad\dfrac{1}{r\mu}\\
        \verb|a|\quad=&\quad\dfrac{j\omega\sigma}{r}\\
        \verb|f|\quad=&\quad J_{i,\varphi}
    \end{aligned}
\end{equation}
