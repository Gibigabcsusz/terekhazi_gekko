\section{Szilágyi Gábor szekciója}
Szeretném én az egyenletek rendezgetését MATLAB-ba tuszakolását megcsinálni.

A feladat megoldásához az $\vec{A}-\Phi, \vec{A}$ formalizmust használjuk, mert az előadáson bemutatott indukciós főzőlapos példa alapján ezzel a megközelítéssel egy jól kezelhető parciális differenciálegyenlet-rendszert kapunk, amit a PDEtool segítségével megoldhatunk.
\begin{align}
    \vec{B} = \rot \vec{A}
\end{align}

A peremfeltételekről a következőket lehet tudni:
\begin{align}
    B_n(\vec{r}) &= 0, \quad \vec{r} \in \Gamma_Z\\
    B_n(\vec{r}) &= 0, \quad \vec{r} \in \Gamma_L \cup \Gamma_T
\end{align}
A $B_n$ normális mágneses indukció a $\Gamma_Z$ peremen a szimmetria miatt lesz 0, a $\Gamma_L \cup \Gamma_T$ távoli peremeken pedig a relatíve nagy távolság miatt.
% \begin{itemize}
    % \item A $\Gamma_Z$ peremeken $\vec{J}_n=0$ és $B_n=0$ a hengerszimmetria miatt.
% \end{itemize}