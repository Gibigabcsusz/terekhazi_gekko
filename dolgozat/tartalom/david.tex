\section{Impedancia kiszámítása $rA_{\varphi}$ értékekből}


A fluxusból tudunk következtetni a tekercs impedanciájára. A tekercs fluxusát
egyszerűen ki lehet számolni a PDE megoldásának eredményeként kapott háromszögekből.\\

Egy menetre számított fluxus számítása:

\begin{align}
    \psi \triangleq \int_{s}^{}  \vec{B} \,ds
    \rightarrow  \psi = \int_{s} rot(\vec{A}) \,ds
    = \oint_L \vec{A} \,dl
\end{align}

Ebből már meg tudjuk határozni, hogy a teljes tekercs fluxusát, amihez tudni kell még a
tekercs menetszámát(N) és a tekercs keresztmetsztét (F) amit a feladat meg adott.

Tekercs Fluxus számítása:
\begin{align}
    \psi = \sum_{k = 1}^{N}  \psi_k = \sum_{k = 1}^{N} \oint_{L_k} \vec{A} \,dl
    = \sum_{k = 1}^{N} 2 \pi r_k A_{\varphi,k}
\end{align}

A tekercs homogenizálásával számítható tekercs fluxus:
\begin{align}
    \psi = \frac{1}{\Delta} \sum_{k=1}^N  2 \pi r_k A_{\varphi,k} \Delta F \approx
    \frac{N}{F} \int_{F} 2 \pi r_k A_{\varphi,k} \,dF
\end{align}

az integrál egyszerűen közelíthatő a PDE megoldásából, a háromszöghálóra felírt
integrál közelítő összeggel.

A kiszámított fluxusból már egyszrűen számítható a tekercs ön és kölcsönös indukciója is.
\begin{align}
    \psi = Z_1 I_1 + Z_2 I_2 \\
    Z_1 = \frac{\psi}{I_1}   \\
    Z_2 = \frac{\psi}{I_2}
\end{align}

