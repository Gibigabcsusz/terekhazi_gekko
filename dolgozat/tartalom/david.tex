\section{Impedancia kiszámítása $rA_{\varphi}$ értékekből}


A fluxusból tudunk következtetni a tekercs impedanciájára. A tekercs fluxusát
egyszerűen ki lehet számolni a PDE megoldásának eredményeként kapott háromszögekből.\\

Egy menetre számított fluxus számítása:

\begin{align}
    \psi \triangleq \int_{s}^{}  \vec{B} \,ds
    \rightarrow  \psi = \int_{s} rot(\vec{A}) \,ds
    = \oint_L \vec{A} \,dl
\end{align}

Ebből már meg tudjuk határozni, hogy a teljes tekercs fluxusát, amihez tudni kell még a
tekercs menetszámát(N) és a tekercs keresztmetsztét (F) amit a feladat meg adott.

Tekercs Fluxus számítása:
\begin{align}
    \psi = \sum_{k = 1}^{N}  \psi_k = \sum_{k = 1}^{N} \oint_{L_k} \vec{A} \,dl
    = \sum_{k = 1}^{N} 2 \pi r_k A_{\varphi,k}
\end{align}

A tekercs homogenizálásával számítható tekercs fluxus:
\begin{align}
    \psi = \frac{1}{\Delta} \sum_{k=1}^N  2 \pi r_k A_{\varphi,k} \Delta F \approx
    \frac{N}{F} \int_{F} 2 \pi r_k A_{\varphi,k} \,dF
\end{align}

az integrál egyszerűen közelíthatő a FEM megoldásából, a háromszöghálóra felírt
integrál közelítő összeggel.

A kiszámított fluxusból már egyszrűen számítható a tekercs impedanciája is.

Kapocs feszültség:
\begin{align}
    U = jw\psi
\end{align}
Tekercs árama:
\begin{align}
    I = J_{\varphi} * \frac{F}{N}
\end{align}
Impedancia:
\begin{align}
    Z = \frac{U}{I} = \frac{jw\psi}{I}
\end{align}

\section{A gömb méret változása és hiánya okozta Impedancia változás}

\subsection{A földet tekintsük szigetelőnek}

peremfeltételek mindenhol : h = 1, r=0\\
Tekercs PDE: c = 1./(4*pi*1e-7)./x ; a = 0.0; f = 1/(0.05*0.02) ; ELIPTIC \\
Levegő PDE : c = 1./x./(4*pi*1e-7) ; a = 0.0; f = 0; ELIPTIC \\
Föld PDE:    c = 1./x./(4*pi*1e-7) ; a = 0.0; f = 0; ELIPTIC \\
Gömb PDE:    c = 1./x./(4*pi*1e-7) ; a = j*2*pi*500*35e6./x; f = 0; ELIPTIC \\

Tekercs impedanciája gömb nélkül:\\
Z0 = 0.0000e+00 + 1.2193e+03i\\

Tekercs impedanciája gömbbel:\\
átmérő: 0.023 \\
távolság: -0.10 \\
Z0 = 7.9215e-02 + 1.2189e+03i\\

\subsection{A földet tekintsük vezetőnek (kezdetleges)}
$\sigma = 1 S/m$\\
peremfeltételek mindenhol : h = 1, r=0\\
Tekercs PDE: c = 1./(4*pi*1e-7)./x ; a = 0.0; f = 1/(0.05*0.02) ; ELIPTIC \\
Levegő PDE : c = 1./x./(4*pi*1e-7) ; a = 0.0; f = 0; ELIPTIC \\
Föld PDE:    c = 1./x./(4*pi*1e-7) ; a = j*2*pi*500./x*1; f = 0; ELIPTIC \\
Gömb PDE:    c = 1./x./(4*pi*1e-7) ; a = j*2*pi*500*35e6./x; f = 0; ELIPTIC \\