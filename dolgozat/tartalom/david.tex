\section{Impedancia kiszámítása $rA_{\varphi}$ értékekből}
%
A fluxusból tudunk következtetni a tekercs impedanciájára. A tekercs fluxusát
egyszerűen ki lehet számolni a PDE megoldásának eredményeként kapott háromszögekből.\\
%
Egy menetre számított fluxus számítása:
%
\begin{align}
    \Psi \triangleq \int_{s}^{}  \vec{B} \,ds
    \rightarrow  \Psi = \int_{s} rot(\vec{A}) \,ds
    = \oint_L \vec{A} \,dl
\end{align}
%
Ebből már meg tudjuk határozni, hogy a teljes tekercs fluxusát, amihez tudni kell még a
tekercs menetszámát(N) és a tekercs keresztmetsztét (F) amit a feladat meg adott.
%
Tekercs Fluxus számítása:
\begin{align}
    \Psi = \sum_{k = 1}^{N}  \Psi_k = \sum_{k = 1}^{N} \oint_{L_k} \vec{A} \,dl
    = \sum_{k = 1}^{N} 2 \pi r_k A_{\varphi,k}
\end{align}
%
A tekercs homogenizálásával számítható tekercs fluxus:
\begin{align}
    \Psi = \frac{1}{\Delta} \sum_{k=1}^N  2 \pi r_k A_{\varphi,k} \Delta F \approx
    \frac{N}{F} \int_{F} 2 \pi r_k A_{\varphi,k} \,dF
\end{align}
%
az integrál egyszerűen közelíthatő a FEM megoldásából, a háromszöghálóra felírt
integrál közelítő összeggel.
%
A kiszámított fluxusból már egyszrűen számítható a tekercs impedanciája is.
%
Kapocsfeszültség:
\begin{align}
    U = jw\Psi
\end{align}
Tekercs árama:
\begin{align}
    I = J_{\varphi} * \frac{F}{N}
\end{align}
Impedancia:
\begin{align}
    Z = \frac{U}{I} = \frac{jw\Psi}{I}
\end{align}
%
\section{A gömb méretének és mélységének változtatása}
%
\subsection{A földet tekintsük szigetelőnek}
%
\begin{align}
    \sigma_t\quad=\quad\qty{0}{\siemens\per\metre}
\end{align}
A pdetool-ban megadott PDE összetevők:\\
\vspace{0.2cm}
\begin{center}
\begin{tabular}{|l||c|c|c|}
    \hline
                    & \verb|c|                 & \verb|a|                   & \verb|f| \\
    \hline
    \hline
    Tekercs      & \verb|1./(4*pi*1e-7)./x| & \verb|0|                   & \verb|2000/(0.05*0.02)| \\
    \hline
    Levegő       & \verb|1./(4*pi*1e-7)./x| & \verb|0|                   & \verb|0| \\
    \hline
    Gömb         & \verb|1./(4*pi*1e-7)./x| & \verb|0|                   & \verb|0| \\
    \hline
    Talaj        & \verb|1./(4*pi*1e-7)./x| & \verb|j*2*pi*500*35e6./x|  & \verb|0| \\
    \hline
\end{tabular}\\
\end{center}
\vspace{0.5cm}
A peremeken mindenhol az \verb|1*u=0| Dirichlet-peremfeltételt adtuk meg. Ez egyben $A_{\varphi}$-re is Dirichlet-peremfeltételt jelent, ami miatt a peremeken $B_n=0$.\\[3ex]
%
Tekercs impedanciája gömb nélkül:
\begin{align}
    Z_0\quad=\quad0 + 1.2157\times10^3j\quad\Omega
\end{align}
%
A szimulációhoz konvergenciavizsgálatot végeztünk, aminél megvizsgáltuk a ritkább háló és a közelibb peremek hatását. A fenti alap esetben a hálót háromszor finomítottuk, illetve a szimulációs tartomány méretparamétere (a bal és jobb oldali perem távolsága) $R=\qty{0.5}{\metre}$ volt. Megvizsgáltuk az alap elrendezést szigetelő talajjal és gömb nélkül, de csak kétszeri hálófinomítás mellett. Így a tekercs impedanciájára a következő adódott:
\begin{align}
    Z_{0,mesh}\quad=&\quad0 + 1.2185\times10^3j\quad\Omega\\
    \dfrac{|Z_{0,mesh}-Z_0|}{|Z_0|}\quad=&\quad0.0023
\end{align}
%
Ezután megvizsgáltuk ugyanazt a kiinduló elrendezést, csak $R=\qty{0.3}{\metre}$ választással. Erre a következő tekercs induktivitás adódott:
\begin{align}
    Z_{0,R}\quad=&\quad0 + 1.2185\times10^3j\quad\Omega\\
    \dfrac{|Z_{0,R}-Z_0|}{|Z_0|}\quad=&\quad0.0048
\end{align}
%
Mindkét relatív hiba $1\%$ alatti, így elfogadhatjuk a részletesebb modellből kapott eredményeket, tehát innentől $R=0.5$ és háromszori hálófinomítást használunk.\\[3ex]
A tekercs impedanciája gömbbel, különböző paraméterekkel $[\Omega]$:
\vspace{0.2cm}
\begin{center}
\begin{tabular}{|c|c|c|c|}
    \hline
    \diagbox{r[m]}{d[m]} & 0.08                     & 0.13                     & 0.18                     \\
    \hline
    \hline
    0.01                 &                          & 5.4765e-03 + 1.2154e+03j &                          \\
    \hline
    0.03                 & 6.7525e-01 + 1.2124e+03j & 8.5124e-02 + 1.2153e+03j & 1.5855e-02 + 1.2156e+03j \\
    \hline
    0.05                 &                          & 3.5212e-01 + 1.2133e+03j &                          \\
    \hline
\end{tabular}\\
\end{center}
\vspace{0.5cm}
%
Innen tehát az impedancia megváltozása ($\Delta Z$) a gömb nálküli esethez képest [$\Omega$]:
\begin{center}
\begin{tabular}{|c|c|c|c|}
    \hline
    \diagbox{r[m]}{d[m]} & 0.08                     & 0.13                     & 0.18                     \\
    \hline
    \hline
    0.01                 &                          & 0.0055 - 0.3430j &                          \\
    \hline
    0.03                 & 0.6753 - 3.3430j & 0.0851 - 0.4430j & 0.0159 - 0.1430i \\
    \hline
    0.05                 &                          & 0.3521 - 2.4430i &                          \\
    \hline
\end{tabular}\\
\end{center}
\vspace{0.5cm}
%
A távolság csökentésével illetve a gömb átmérő csökkentésével csökken az impedancia reális
része míg a komplex szinte változatlan marad. Azaz a tekercsnek a gömb megjelenésével
lessz ellenállása is. Az ellenállás  nagysága pedig attól függ, hogy milyen távol van a gömb
a tekercstől és a gömb méretétől.
%
\subsection{A földet tekintsük rossz vezetőnek}
\begin{align}
    \sigma_t\quad=\quad\qty{1}{\siemens\per\metre}
\end{align}
A pdetool-ban megadott PDE összetevők:\\
\vspace{0.2cm}
\begin{center}
\begin{tabular}{|l||c|c|c|}
    \hline
                    & \verb|c|                 & \verb|a|                   & \verb|f| \\
    \hline
    \hline
    Tekercs      & \verb|1./(4*pi*1e-7)./x| & \verb|0|                   & \verb|1/(0.05*0.02)| \\
    \hline
    Levegő       & \verb|1./(4*pi*1e-7)./x| & \verb|0|                   & \verb|0| \\
    \hline
    Gömb         & \verb|1./(4*pi*1e-7)./x| & \verb|j*2*pi*500./x*1|     & \verb|0| \\
    \hline
    Talaj        & \verb|1./(4*pi*1e-7)./x| & \verb|j*2*pi*500*35e6./x|  & \verb|0| \\
    \hline
\end{tabular}\\
\end{center}
\vspace{0.5cm}
A peremeken mindenhol az \verb|1*u=0| Dirichlet-peremfeltételt adtuk meg.\\[3ex]
%
A tekercs impedanciája gömb nélkül: % + 
\begin{align}
    Z_0\quad=\quad6.6340\times10^{-4} + 1.2157\times10^3j \quad\Omega
\end{align}
A tekercs impedanciája gömbbel, különböző paraméterekkel $[\Omega]$:
%
\vspace{0.2cm}
\begin{center}
\begin{tabular}{|c|c|c|c|}
    \hline
    \diagbox{r[m]}{d[m]} & 0.08                     & 0.13                     & 0.18                     \\
    \hline
    \hline
    0.01                 &                          & 6.1399e-03 + 1.2154e+03j &                          \\
    \hline
    0.03                 & 6.7587e-01 + 1.2124e+03j & 8.5774e-02 + 1.2153e+03j & 1.6515e-02 + 1.2156e+03j \\
    \hline
    0.05                 &                          & 3.5271e-01 + 1.2133e+03j &                          \\
    \hline
\end{tabular}\\
\end{center}
\vspace{0.5cm}
%
Innen tehát az impedancia megváltozása ($\Delta Z$) a gömb nálküli esethez képest [$\Omega$]:
\vspace{0.2cm}
\begin{center}
\begin{tabular}{|c|c|c|c|}
    \hline
    \diagbox{r[m]}{d[m]} & 0.08                     & 0.13                     & 0.18                     \\
    \hline
    \hline
    0.01                 &                          & 0.0055 - 0.3430j &                          \\
    \hline
    0.03                 & 0.6752 - 3.3430j & 0.0851 - 0.4430j & 0.0159 - 0.1430j \\
    \hline
    0.05                 &                          & 0.3520 - 2.4430j &                          \\
    \hline
\end{tabular}\\
\end{center}
\vspace{0.5cm}
%
A tapasztalat szinte ugyanaz, mint amikor a földet szigetelőnek tekintjük, azzal a különbséggel, hogy a már a gömb nélkül is van valós része az impedanciának. Ez azért logikus, mert a véges vezetőképességű talajban létrejövő örvényáramok miatti ohmos veszteség jelenik meg a tekercs impedanciájának valós részeként. 
\subsection{A földet tekintsük jó vezetőnek}
A talaj vezetőképességét a volframéval megegyezőnek választottuk, mert az körülbelül fele a gömbre vonatkozónak, ami pedig az alumíniuménak felel meg.
\begin{align}
    \sigma_t\quad=\quad\qty{1.82e7}{\siemens\per\metre}
\end{align}
A pdetool-ban megadott PDE összetevők:\\
\vspace{0.2cm}
\begin{center}
\begin{tabular}{|l||c|c|c|}
    \hline
                    & \verb|c|                 & \verb|a|                   & \verb|f| \\
    \hline
    \hline
    Tekercs      & \verb|1./(4*pi*1e-7)./x| & \verb|0|                   & \verb|1/(0.05*0.02)| \\
    \hline
    Levegő       & \verb|1./(4*pi*1e-7)./x| & \verb|0|                   & \verb|0| \\
    \hline
    Gömb         & \verb|1./(4*pi*1e-7)./x| & \verb|j*2*pi*500./x*1.82e7|     & \verb|0| \\
    \hline
    Talaj        & \verb|1./(4*pi*1e-7)./x| & \verb|j*2*pi*500*35e6./x|  & \verb|0| \\
    \hline
\end{tabular}\\
\end{center}
\vspace{0.5cm}
A peremeken mindenhol az \verb|1*u=0| Dirichlet-peremfeltételt adtuk meg.\\[3ex]
%
A tekercs impedanciája gömb nélkül: % + 
\begin{align}
    Z_0\quad=\quad1.3793\times10^{1} + 1.0905\times10^3j \quad\Omega
\end{align}
A tekercs impedanciája gömbbel, különböző paraméterekkel $[\Omega]$:
%
\vspace{0.2cm}
\begin{center}
\begin{tabular}{|c|c|c|c|}
    \hline
    \diagbox{r[m]}{d[m]} & 0.08                     & 0.13                     & 0.18                     \\
    \hline
    \hline
    0.01                 &                          & 1.3922e+01 + 1.0902e+03j &                          \\
    \hline
    0.03                 & 1.3925e+01 + 1.0900e+03j & 1.3953e+01 + 1.0903e+03j & 1.3872e+01 + 1.0904e+03j \\
    \hline
    0.05                 &                          & 1.4064e+01 + 1.0901e+03j &                          \\
    \hline
\end{tabular}\\
\end{center}
\vspace{0.5cm}
%
Innen tehát az impedancia megváltozása ($\Delta Z$) a gömb nálküli esethez képest [$\Omega$]:
\vspace{0.2cm}
\begin{center}
\begin{tabular}{|c|c|c|c|}
    \hline
    \diagbox{r[m]}{d[m]} & 0.08                     & 0.13                     & 0.18                     \\
    \hline
    \hline
    0.01                 &                          & 0.1292 - 0.2737i &                          \\
    \hline
    0.03                 & 0.1322 - 0.4737i & 0.1602 - 0.1737j & 0.0792 - 0.0737i \\
    \hline
    0.05                 &                          & 0.2712 - 0.3737j &                          \\
    \hline
\end{tabular}\\
\end{center}
\vspace{0.5cm}
%
Az eredményekből arra lehet következtetni, hogy ha elég jó vezetőképességű a talaj, akkor az eltemetett jobb vezetőképességű gömb miatti impedancia megváltozás kisebb, mint a szigetelő vagy rossz vezetőképességű talaj esetén. Ha a talaj rossz vezető, akkor a gömb hatására megjelenő veszteség felhasználható detekcióra egy nagy jósági tényezőjű rezgőkör jósági tényezőjének lerontására, ami detektálható. Ellenben jó vezető talajnál már a tekercs impedanciájának valós része a gömb nélkül is számottevő, így a detekció sokkal nehezebb.