\section{Impedancia kiszámítása $rA_{\varphi}$ értékekből}
%
A fluxusból tudunk következtetni a tekercs impedanciájára. A tekercs fluxusát
egyszerűen ki lehet számolni a PDE megoldásának eredményeként kapott háromszögekből.\\
%
Egy menetre számított fluxus számítása:
%
\begin{align}
    \Psi \triangleq \int_{s}^{}  \vec{B} \,ds
    \rightarrow  \Psi = \int_{s} rot(\vec{A}) \,ds
    = \oint_L \vec{A} \,dl
\end{align}
%
Ebből már meg tudjuk határozni, hogy a teljes tekercs fluxusát, amihez tudni kell még a
tekercs menetszámát(N) és a tekercs keresztmetsztét (F) amit a feladat meg adott.
%
Tekercs Fluxus számítása:
\begin{align}
    \Psi = \sum_{k = 1}^{N}  \Psi_k = \sum_{k = 1}^{N} \oint_{L_k} \vec{A} \,dl
    = \sum_{k = 1}^{N} 2 \pi r_k A_{\varphi,k}
\end{align}
%
A tekercs homogenizálásával számítható tekercs fluxus:
\begin{align}
    \Psi = \frac{1}{\Delta} \sum_{k=1}^N  2 \pi r_k A_{\varphi,k} \Delta F \approx
    \frac{N}{F} \int_{F} 2 \pi r_k A_{\varphi,k} \,dF
\end{align}
%
az integrál egyszerűen közelíthatő a FEM megoldásából, a háromszöghálóra felírt
integrál közelítő összeggel.
%
A kiszámított fluxusból már egyszrűen számítható a tekercs impedanciája is.
%
Kapocsfeszültség:
\begin{align}
    U = jw\Psi
\end{align}
Tekercs árama:
\begin{align}
    I = J_{\varphi} * \frac{F}{N}
\end{align}
Impedancia:
\begin{align}
    Z = \frac{U}{I} = \frac{jw\Psi}{I}
\end{align}
%
\section{A gömb méretének és mélységének változtatása}
%
\subsection{A földet tekintsük szigetelőnek}
%
A pdetool-ban megadott PDE összetevők:\\
\vspace{0.2cm}
\begin{center}
\begin{tabular}{|l||c|c|c|}
    \hline
                    & \verb|c|                 & \verb|a|                   & \verb|f| \\
    \hline
    \hline
    Tekercs      & \verb|1./(4*pi*1e-7)./x| & \verb|0|                   & \verb|1/(0.05*0.02)| \\
    \hline
    Levegő       & \verb|1./(4*pi*1e-7)./x| & \verb|0|                   & \verb|0| \\
    \hline
    Gömb         & \verb|1./(4*pi*1e-7)./x| & \verb|0|                   & \verb|0| \\
    \hline
    Talaj        & \verb|1./(4*pi*1e-7)./x| & \verb|j*2*pi*500*35e6./x|  & \verb|0| \\
    \hline
\end{tabular}\\
\end{center}
\vspace{0.5cm}
A peremeken mindenhol az \verb|1*u=0| Dirichlet-peremfeltételt adtuk meg.
%
Tekercs impedanciája gömb nélkül:
\begin{align}
    Z_0 = 0 + 1.2193\times10^3j \quad\Omega
\end{align}
%
A tekercs impedanciája gömbbel, különböző paraméterekkel $[\Omega]$:
\vspace{0.2cm}
\begin{center}
\begin{tabular}{|c|c|c|c|}
    \hline
    \diagbox{r[m]}{d[m]} & 0.08                     & 0.13                     & 0.18                     \\
    \hline
    \hline
    0.01                 &                          & 5.4073e-03 + 1.2192e+03j &                          \\
    \hline
    0.03                 & 6.5651e-01 + 1.2163e+03j & 7.9215e-02 + 1.2189e+03j & 1.4745e-02 + 1.2192e+03j \\
    \hline
    0.05                 &                          & 3.1053e-01 + 1.2171e+03j &                          \\
    \hline
\end{tabular}\\
\end{center}
\vspace{0.5cm}
%
A távolság csökentésével illetve a gömb átmérő csökkentésével csökken az impedancia reális
része míg a komplex szinte változatlan marad. Azaz a tekercsnek a gömb megjelenésével
lessz ellenállása is. Az ellenállás  nagysága pedig attól függ, hogy milyen távol van a gömb
a tekercstől és a gömb méretétől.
%
\subsection{A földet tekintsük rossz vezetőnek}
\begin{align}
    \sigma = 1 S/m
\end{align}
A pdetool-ban megadott PDE összetevők:\\
\vspace{0.2cm}
\begin{center}
\begin{tabular}{|l||c|c|c|}
    \hline
                    & \verb|c|                 & \verb|a|                   & \verb|f| \\
    \hline
    \hline
    Tekercs      & \verb|1./(4*pi*1e-7)./x| & \verb|0|                   & \verb|1/(0.05*0.02)| \\
    \hline
    Levegő       & \verb|1./(4*pi*1e-7)./x| & \verb|0|                   & \verb|0| \\
    \hline
    Gömb         & \verb|1./(4*pi*1e-7)./x| & \verb|j*2*pi*500./x*1|     & \verb|0| \\
    \hline
    Talaj        & \verb|1./(4*pi*1e-7)./x| & \verb|j*2*pi*500*35e6./x|  & \verb|0| \\
    \hline
\end{tabular}\\
\end{center}
\vspace{0.5cm}
A peremeken mindenhol az \verb|1*u=0| Dirichlet-peremfeltételt adtuk meg.
%
A tekercs impedanciája gömb nélkül: % + 
\begin{align}
    Z_0 = 6.6401\times10^{-4} + 1.2193\times10^3j \quad\Omega
\end{align}
A tekercs impedanciája gömbbel, különböző paraméterekkel $[\Omega]$:
%
\vspace{0.2cm}
\begin{center}
\begin{tabular}{|c|c|c|c|}
    \hline
    \diagbox{r[m]}{d[m]} & 0.08                     & 0.13                     & 0.18                     \\
    \hline
    \hline
    0.01                 &                          & 6.0710e-03 + 1.2192e+03j &                          \\
    \hline
    0.03                 & 6.5713e-01 + 1.2163e+03j & 7.9866e-02 + 1.2189e+03j & 1.5405e-02 + 1.2192e+03j \\
    \hline
    0.05                 &                          & 3.1112e-01 + 1.2171e+03j &                          \\
    \hline
\end{tabular}\\
\end{center}
\vspace{0.5cm}
%
A tapasztalat szinte ugyanaz, mint amikor a földet szigetelőnek tekintjük, azzal a különbséggel, hogy a már a gömb nélkül is van valós része az impedanciának.